\documentclass[a4paper]{article}

\usepackage[english]{babel}
\usepackage[utf8]{inputenc}
\usepackage{amsmath}
\usepackage{graphicx}
\usepackage[colorinlistoftodos]{todonotes}
\usepackage{algorithm}
\usepackage[noend]{algpseudocode}

\usepackage{listings, lstautogobble}
\lstset{language=SQL,
                basicstyle=\footnotesize\ttfamily,
                keywordstyle=\footnotesize\color{blue}\ttfamily,
		breaklines=true,
		frame=single,
		autogobble=true,
		showstringspaces=false
}
\definecolor{light-gray}{gray}{0.95}
% \newcommand{\code}[1]{\colorbox{light-gray}\allowbreak{\texttt{#1}}}
\newcommand{\code}[1]{\allowbreak{\texttt{#1}}}


\title{Praktikum 4.3}

\author{Praktikum Basis Data\\ Program Studi Manajemen Informatika \\ FMIPA, Universitas Riau }

% \date{\today}
\date{May 27, 2022}

\begin{document}
\maketitle

\section{Pengantar}%
\label{sec:pengantar}

Pada praktikum 4.3 ini, kita akan mengenal dan mempelajari tentang konsep \textit{Refrential Integrity Constraint} /  \textit{Behavior Foreign Key} di dalam SQL (MySQL).

\section{Refrential Integrity Constraint}%
\label{sec:section_name}

\textit{Refrential Integrity Constraint} merupakan aturan yang digunakan untuk menjaga kualitas dari informasi yang disimpan di \textit{database}. \textit{Refrential  Integrity Constraint} akan membantu kita untuk memastikan untuk tidak terjadinya kerusakan atau hilangnya \textit{integritas data} saat proses \textit{Insert}, \textit{Update}, dan \textit{Delete}.

Misalnya ketika suatu tabel berelasi dengan tabel lain dan kita menghapus salah satu data yang berelasi, MySQL akan menolak operasi delete tersebut. Hal itu tidak bisa dilakukan karena statusnya \textit{restrict} (nanti kita akan bahas).  Kita juga bisa mengubah fitur ini jika kita mau, ada banyak hal yang bisa dilakukan ketika data berelasi dihapus, defaultnya memang akan ditolak (RESTRICT)

Untuk menggunakan fitur \textit{Referential Integrity} tipe dari \textit{store engine} tabel yang digunakan adalah yang support transaksional, salah satu contoh di MySQL yang support transaksional adalah \textit{store engine InnoDB}..

\section{Jenis-Jenis Refrential Integrity Constraint / Behavior Foreign Key}%
\label{sec:section_name}
\begin{figure}[htpb]
	\centering
	\includegraphics[width=0.8\linewidth]{pic/behavior-fk.png}
\end{figure}


\begin{enumerate}
	\item Restrict	

\textit{Restrict} dalam bahasa indonesia adalah membatasi, maksudnya adalah data pada tabel induk tidak bisa di \textit{delete} atau di \textit{update} (reference key nya) bila data tersebut memiliki relasi pada tabel lain.

\begin{lstlisting}
CREATE TABLE products (
    id INT NOT NULL,
    name VARCHAR(100) NOT NULL,
    price INT(11),
    PRIMARY KEY(id)
) ENGINE=INNODB;

CREATE TABLE customers (
    id INT NOT NULL,
    name VARCHAR(100) NOT NULL,
    PRIMARY KEY (id)
) ENGINE=INNODB;

CREATE TABLE product_order (
    id INT NOT NULL AUTO_INCREMENT,
    product_id INT NOT NULL,
    customer_id INT NOT NULL,
    PRIMARY KEY(id)
) ENGINE=INNODB;

ALTER TABLE product_order 
	ADD CONSTRAINT fk_po_customers 
	FOREIGN KEY (customer_id) REFERENCES customers(id)
	ON DELETE RESTRICT ON UPDATE RESTRICT;
ALTER TABLE product_order 
	ADD CONSTRAINT fk_po_products
	FOREIGN KEY (product_id) REFERENCES products(id)
	ON DELETE RESTRICT ON UPDATE RESTRICT;
\end{lstlisting}

\textit{Restrict} adalah sintak yang digunakan untuk menambahkan \textit{refrential integrity constraint} dengan tipe \textit{restrict}. Sekarang kita coba \textit{insert} data kedalam tabel tersebut

\begin{lstlisting}
INSERT INTO products(id,name,price) VALUES (1,'Sunlight',15000);
INSERT INTO customers(id,name) VALUES (1,'Vivianika Nathania Harsan');
INSERT INTO product_order(product_id,customer_id) VALUES(1,1);
\end{lstlisting}

Kemudian kita akan coba  \textit{update} (bukan update barisnya ya, tapi \textit{primary key reference}) dan \textit{delete} data yang ada pada tabel parent (\textit{products} dan \textit{customers}). Hasilnya:

\begin{lstlisting}
MariaDB [testing]> UPDATE products SET id=5 WHERE id=1;
ERROR 1451 (23000): Cannot delete or update a parent row: a foreign key constraint fails (`testing`.`product_order`, CONSTRAINT `product_order_ibfk_1` FOREIGN KEY (`product_id`) REFERENCES `products` (`id`))

MariaDB [testing]> DELETE FROM customers WHERE id=1;
ERROR 1451 (23000): Cannot delete or update a parent row: a foreign key constraint fails (`testing`.`product_order`, CONSTRAINT `product_order_ibfk_2` FOREIGN KEY (`customer_id`) REFERENCES `customers` (`id`))

\end{lstlisting}

	\item Cascade

\textit{Cascade} dalam bahasa Indonesia adalah bertingkat, maksudnya adalah bila data pada tabel induk di \textit{delete} atau di \textit{update} maka secara otomatis data pada tabel lain yang \code{memiliki relasi} akan di \textit{delete} atau di \textit{update} juga.

Sekarang kita ganti behavior \textit{update} dan \textit{delete} dari table sebelumnya menjadi  \code{Cascade}
\begin{lstlisting}
ALTER TABLE product_order DROP FOREIGN KEY fk_po_customers;
ALTER TABLE product_order DROP FOREIGN KEY fk_po_products;

ALTER TABLE product_order
	ADD CONSTRAINT fk_po_customers 
	FOREIGN KEY (customer_id) REFERENCES customers(id)
	ON DELETE CASCADE ON UPDATE CASCADE;
ALTER TABLE product_order 
	ADD CONSTRAINT fk_po_products
	FOREIGN KEY (product_id) REFERENCES products(id)
	ON DELETE CASCADE ON UPDATE CASCADE;
\end{lstlisting}

Kemudian kita akan coba  \textit{update} (bukan update barisnya ya, tapi \textit{primary key reference}) dan \textit{delete} data yang ada pada tabel parent (\textit{products} dan \textit{customers}). Hasilnya:

\begin{lstlisting}
MariaDB [testing]> UPDATE products SET id=5 WHERE id=1;
Query OK, 1 row affected (0.050 sec)
Rows matched: 1  Changed: 1  Warnings: 0

MariaDB [testing]> SELECT * FROM product_order;
+----+------------+-------------+
| id | product_id | customer_id |
+----+------------+-------------+
|  1 |          5 |           1 |
+----+------------+-------------+
1 row in set (0.001 sec)


MariaDB [testing]> DELETE FROM customers WHERE id=1;
Query OK, 1 row affected (0.048 sec)

MariaDB [testing]> SELECT * FROM product_order;
Empty set (0.001 sec)

\end{lstlisting}

	\item No Action

\textit{No Action} ini setara dengan RESTRICT. MySQL menolak operasi delete atau update untuk tabel induk jika ada referensi dengan tabel lainnya. 

Sekarang kita ganti behavior \textit{update} dan \textit{delete} dari table sebelumnya menjadi \code{No Action}
\begin{lstlisting}

ALTER TABLE product_order DROP FOREIGN KEY fk_po_customers;
ALTER TABLE product_order DROP FOREIGN KEY fk_po_products;

ALTER TABLE product_order
	ADD CONSTRAINT fk_po_customers 
	FOREIGN KEY (customer_id) REFERENCES customers(id)
	ON DELETE NO ACTION ON UPDATE NO ACTION;
ALTER TABLE product_order 
	ADD CONSTRAINT fk_po_products
	FOREIGN KEY (product_id) REFERENCES products(id)
	ON DELETE NO ACTION ON UPDATE NO ACTION;
\end{lstlisting}

Kemudian kita akan coba  \textit{update} (bukan update barisnya ya, tapi \textit{primary key reference}) dan \textit{delete} data yang ada pada tabel parent (\textit{products} dan \textit{customers}). Hasilnya:

\begin{lstlisting}
MariaDB [testing]> UPDATE customers SET id=4 WHERE id=1;
ERROR 1451 (23000): Cannot delete or update a parent row: a foreign key constraint fails (`testing`.`product_order`, CONSTRAINT `fk_po_customers` FOREIGN KEY (`customer_id`) REFERENCES `customers` (`id`) ON DELETE NO ACTION ON UPDATE NO ACTION)
MariaDB [testing]> DELETE FROM products WHERE id=1;
ERROR 1451 (23000): Cannot delete or update a parent row: a foreign key constraint fails (`testing`.`product_order`, CONSTRAINT `fk_po_products` FOREIGN KEY (`product_id`) REFERENCES `products` (`id`) ON DELETE NO ACTION ON UPDATE NO ACTION)
MariaDB [testing]> 
\end{lstlisting}


	\item Set Null

\textit{Set null} adalah menghapus atau meng-update baris data dalam tabel induk dan men-set kolom atau beberapa kolom dalam tabel anak menjadi \textit{null}. hal ini akan valid jika kolom \textit{foreign key} tidak \textit{not null}. 

Sekarang kita ganti behavior \textit{update} dan \textit{delete} dari table sebelumnya menjadi \code{SET NULL}
\begin{lstlisting}
DROP TABLE product_order;

CREATE TABLE product_order (
    id INT NOT NULL AUTO_INCREMENT,
    product_id INT,
    customer_id INT,
    PRIMARY KEY(id)
) ENGINE=INNODB;

ALTER TABLE product_order
	ADD CONSTRAINT fk_po_customers 
	FOREIGN KEY (customer_id) REFERENCES customers(id)
	ON DELETE SET NULL ON UPDATE SET NULL;
ALTER TABLE product_order 
	ADD CONSTRAINT fk_po_products
	FOREIGN KEY (product_id) REFERENCES products(id)
	ON DELETE SET NULL ON UPDATE SET NULL;

INSERT INTO product_order(product_id,customer_id) VALUES(1,1);
\end{lstlisting}

Kemudian kita akan coba  \textit{update} (bukan update barisnya ya, tapi \textit{primary key reference}) dan \textit{delete} data yang ada pada tabel parent (\textit{products} dan \textit{customers}). Hasilnya:

\begin{lstlisting}
MariaDB [testing]> SELECT * FROM product_order;
+----+------------+-------------+
| id | product_id | customer_id |
+----+------------+-------------+
|  1 |          1 |           1 |
+----+------------+-------------+
1 row in set (0.001 sec)

MariaDB [testing]> UPDATE products SET id=5 where id=1;
Query OK, 1 row affected (0.045 sec)
Rows matched: 1  Changed: 1  Warnings: 0

MariaDB [testing]> SELECT * FROM product_order;
+----+------------+-------------+
| id | product_id | customer_id |
+----+------------+-------------+
|  1 |       NULL |           1 |
+----+------------+-------------+
1 row in set (0.001 sec)

MariaDB [testing]> DELETE FROM customers WHERE id=1;
Query OK, 1 row affected (0.056 sec)

MariaDB [testing]> SELECT * FROM product_order;
+----+------------+-------------+
| id | product_id | customer_id |
+----+------------+-------------+
|  1 |       NULL |        NULL |
+----+------------+-------------+
1 row in set (0.001 sec)
\end{lstlisting}

\end{enumerate}
\end{document}
