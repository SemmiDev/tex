\documentclass[a4paper]{article}

\usepackage[english]{babel}
\usepackage[utf8]{inputenc}
\usepackage{amsmath}
\usepackage{graphicx}
\usepackage[colorinlistoftodos]{todonotes}
\usepackage{algorithm}
\usepackage[noend]{algpseudocode}

\usepackage{listings, lstautogobble}
\lstset{language=SQL,
                basicstyle=\footnotesize\ttfamily,
                keywordstyle=\footnotesize\color{blue}\ttfamily,
		breaklines=true,
		frame=single,
		autogobble=true,
		showstringspaces=false
}
\definecolor{light-gray}{gray}{0.95}
% \newcommand{\code}[1]{\colorbox{light-gray}\allowbreak{\texttt{#1}}}
\newcommand{\code}[1]{\allowbreak{\texttt{#1}}}


\title{Praktikum 4.4}

\author{Praktikum Basis Data\\ Program Studi Manajemen Informatika \\ FMIPA, Universitas Riau }

% \date{\today}
\date{June 10, 2022}

\begin{document}
\maketitle

\section{Pengantar}%
\label{sec:pengantar}


Pada praktikum 4.4 ini, kita akan mengenal dan mempelajari tentang konsep \textit{View}  di dalam SQL (MySQL).

\section{Pengenalan View}%

Di dalam MySQL, View dapat didefenisikan sebagai \textbf{tabel virtual}. Tabel ini bisa berasal dari tabel lain, atau gabungan dari beberapa tabel. Tujuan dari pembuatan \textit{View} adalah untuk kenyamanan (mempermudah penulisan \textit{query}), untuk keamanan (menyembunyikan beberapa kolom yang bersifat rahasia), atau dalam beberapa kasus bisa digunakan untuk mempercepat proses menampilkan data (terutama jika kita akan menjalankan \textit{query} tersebut secara berulang).

\label{sec:section_name}

\section{Create View}%

Statemen \code{CREATE VIEW} membuat view baru di dalam database. Berikut ini adalah sintaks dasar statemen \code{CREATE VIEW}:
\begin{lstlisting}
	CREATE [OR REPLACE] VIEW [db_name.]view_name [(column_list)]
	AS
	  select-statement;
\end{lstlisting}

Sekarang kita coba buat sampel tabel sederhana:

\begin{lstlisting}
CREATE DATABASE belajar_view;
USE belajar_view;

CREATE TABLE `customers` (
  `customerNumber` int(11) NOT NULL,
  `customerName` varchar(50) NOT NULL,
  `contactLastName` varchar(50) NOT NULL,
  `contactFirstName` varchar(50) NOT NULL,
  `phone` varchar(50) NOT NULL,
  `addressLine1` varchar(50) NOT NULL,
  `addressLine2` varchar(50) DEFAULT NULL,
  `city` varchar(50) NOT NULL,
  `state` varchar(50) DEFAULT NULL,
  `postalCode` varchar(15) DEFAULT NULL,
  `country` varchar(50) NOT NULL,
  `creditLimit` decimal(10,2) DEFAULT NULL,
  PRIMARY KEY (`customerNumber`)
) ENGINE=InnoDB;

CREATE TABLE `payments` (
  `customerNumber` int(11) NOT NULL,
  `checkNumber` varchar(50) NOT NULL,
  `paymentDate` date NOT NULL,
  `amount` decimal(10,2) NOT NULL,
  PRIMARY KEY (`customerNumber`,`checkNumber`),
  CONSTRAINT `payments_ibfk_1` FOREIGN KEY (`customerNumber`) REFERENCES `customers` (`customerNumber`)
) ENGINE=InnoDB;


insert  into `customers`(`customerNumber`,`customerName`,`contactLastName`,`contactFirstName`,`phone`,`addressLine1`,`addressLine2`,`city`,`state`,`postalCode`,`country`,`creditLimit`) values 
(1,'Atelier graphique','Schmitt','Carine ','40.32.2555','54, rue Royale',NULL,'Nantes',NULL,'44000','France','21000.00'),
(2,'Signal Gift Stores','King','Jean','7025551838','8489 Strong St.',NULL,'Las Vegas','NV','83030','USA','71800.00');


insert  into `payments`(`customerNumber`,`checkNumber`,`paymentDate`,`amount`) values 
(1,'HQ336336','2004-10-19','6066.78'),
(1,'JM555205','2003-06-05','14571.44'),
(2,'OM314933','2004-12-18','1676.14'),
(2,'BO864823','2004-12-17','14191.12');
\end{lstlisting}

Kemudian kita kueri untuk mengambil data dari tabel \textit{customers} dan \textit{payments} menggunakan \textbf{INNER JOIN}:

\begin{lstlisting}
SELECT 
    customerName, 
    checkNumber, 
    paymentDate, 
    amount
FROM
    customers
INNER JOIN
    payments USING (customerNumber);
\end{lstlisting}

Maka akan menampilkan hasil berikut:
\begin{lstlisting}
+--------------------+-------------+-------------+----------+
| customerName       | checkNumber | paymentDate | amount   |
+--------------------+-------------+-------------+----------+
| Atelier graphique  | HQ336336    | 2004-10-19  |  6066.78 |
| Atelier graphique  | JM555205    | 2003-06-05  | 14571.44 |
| Signal Gift Stores | BO864823    | 2004-12-17  | 14191.12 |
| Signal Gift Stores | OM314933    | 2004-12-18  |  1676.14 |
+--------------------+-------------+-------------+----------+
\end{lstlisting}

Jika suatu saat kita ingin mendapatkan informasi yang sama termasuk \textit{customerName, checkNumber, paymentDate, dan amount}, kita perlu menggunakan kueri yang sama lagi.

Cara yang lebih direkomendasikan untuk melakukannya adalah dengan menyimpan kueri di database dan menetapkan nama untuk kueri tersebut. Inilah yang disebut dengan \textit{View} / tampilan.

Untuk membuat \textit{View} kita menggunakan statemen CREATE VIEW. Statemen ini membuat View \textit{customerPayments} berdasarkan kueri di atas:

\begin{lstlisting}
CREATE VIEW customerPayments
AS 
SELECT 
    customerName, 
    checkNumber, 
    paymentDate, 
    amount
FROM
    customers
INNER JOIN
    payments USING (customerNumber);
\end{lstlisting}


Setelah kita menjalankan pernyataan CREATE VIEW, MySQL akan membuat \textit{View} dan menyimpannya dalam database. Sekarang, Kita dapat mereferensikan \textit{view} sebagai tabel dalam statemen SQL. Misalnya, kita dapat meminta data dari \textit{customerPayments} menggunakan statement SELECT:

\begin{lstlisting}
SELECT * FROM customerPayments;
\end{lstlisting}


Maka akan menampilkan hasil berikut:
\begin{lstlisting}
+--------------------+-------------+-------------+----------+
| customerName       | checkNumber | paymentDate | amount   |
+--------------------+-------------+-------------+----------+
| Atelier graphique  | HQ336336    | 2004-10-19  |  6066.78 |
| Atelier graphique  | JM555205    | 2003-06-05  | 14571.44 |
| Signal Gift Stores | BO864823    | 2004-12-17  | 14191.12 |
| Signal Gift Stores | OM314933    | 2004-12-18  |  1676.14 |
+--------------------+-------------+-------------+----------+
\end{lstlisting}

\section{Drop Views}%
\label{sec:section_name}

\section{Show Views}%
\label{sec:show_views}

\section{Rename Views}%
\label{sec:rename_views}


\end{document}
