\documentclass[a4paper]{article}

\usepackage[english]{babel}
\usepackage[utf8]{inputenc}
\usepackage{amsmath}
\usepackage{graphicx}
\usepackage[colorinlistoftodos]{todonotes}
\usepackage{algorithm}
\usepackage[noend]{algpseudocode}

\usepackage{listings, lstautogobble}
\lstset{language=SQL,
                basicstyle=\footnotesize\ttfamily,
                keywordstyle=\footnotesize\color{blue}\ttfamily,
		breaklines=true,
		frame=single,
		autogobble=true,
		showstringspaces=false
}
\definecolor{light-gray}{gray}{0.95}
% \newcommand{\code}[1]{\colorbox{light-gray}\allowbreak{\texttt{#1}}}
\newcommand{\code}[1]{\allowbreak{\texttt{#1}}}


\title{Praktikum 4.4}

\author{Praktikum Basis Data\\ Program Studi Manajemen Informatika \\ FMIPA, Universitas Riau }

% \date{\today}
\date{June 10, 2022}

\begin{document}
\maketitle

\section{Pengantar}%
\label{sec:pengantar}


Pada praktikum 4.4 ini, kita akan mengenal dan mempelajari tentang konsep \textit{View}  di dalam SQL (MySQL).

\section{Pengenalan View}%

Di dalam MySQL, View dapat didefenisikan sebagai \textbf{tabel virtual}. Tabel ini bisa berasal dari tabel lain, atau gabungan dari beberapa tabel. Tujuan dari pembuatan \textit{View} adalah untuk kenyamanan (mempermudah penulisan \textit{query}), untuk keamanan (menyembunyikan beberapa kolom yang bersifat rahasia), atau dalam beberapa kasus bisa digunakan untuk mempercepat proses menampilkan data (terutama jika kita akan menjalankan query tersebut secara berulang).

\label{sec:section_name}

\section{Create View}%

Statemen \code{CREATE VIEW} membuat view baru di dalam database. Berikut ini adalah sintaks dasar statemen \code{CREATE VIEW}:
\begin{lstlisting}
	CREATE [OR REPLACE] VIEW [db_name.]view_name [(column_list)]
	AS
	  select-statement;
\end{lstlisting}

\section{Drop Views}%
\label{sec:section_name}

\section{Show Views}%
\label{sec:show_views}

\section{Rename Views}%
\label{sec:rename_views}


\end{document}
